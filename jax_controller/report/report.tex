% Setup and frequently used packages
\documentclass[12pt]{article}
\usepackage[a4paper, margin=1in]{geometry}
\usepackage[utf8]{inputenc}
\usepackage[T1]{fontenc}
\usepackage{parskip} % Remove indent of new paragraph
\usepackage{titling}
\usepackage{graphicx} % To include graphics such as pictures
\usepackage{amsmath}
\usepackage{caption} % Used to allow empty captions for figures

% Setup title/author/date
\title{Jax Controller Report}
\author{Group 128}
\date{}

% References
\usepackage{hyperref} % Correct formatting for urls in references
\usepackage[style=apa]{biblatex}
\addbibresource{references.bib}

% Header
\usepackage{fancyhdr}
\pagestyle{fancyplain}
\fancyhf{}
\lhead{\theauthor}
\chead{}
\rhead{\thepage}
\renewcommand{\headrulewidth}{0.4pt}
\renewcommand{\plainheadrulewidth}{0.4pt}


\begin{document}
\maketitle

\section*{Conclusion}

Initial weight parameters was set using Python's random library. The weights were set to a random number within the interval $[0,1)$.


\begin{verbatim}
Training set MSE: 0.0392
Test set MSE: 0.0399
\end{verbatim}

The result above was obtained with a learning rate of 0.001 and 1000 epochs. 

\begin{verbatim}
Training set MSE: 0.0504
Test set MSE: 0.0517
\end{verbatim}

The result above was obtained after setting the amount of epochs to 100 instead of 1000. Still using a learning rate of 0.001.

\begin{verbatim}
Training set MSE: 0.0389
Test set MSE: 0.0383
\end{verbatim}

The best results was obtained with a learning rate of 0.05 and 1000 epochs. Slightly better than a learning rate of
0.001

\section*{Lotka-Volterra Population Plant}



\printbibliography

\end{document}